\documentclass[11pt,a4paper]{article}
\usepackage[utf8]{inputenc}
\usepackage{amsmath,amssymb,amsfonts}
\usepackage{graphicx}
\usepackage{subcaption}
\usepackage{hyperref}
\usepackage[margin=1in]{geometry}
\usepackage{listings}
\usepackage{xcolor}

% Configure code listings
\lstset{
    language=,
    basicstyle=\ttfamily\small,
    keywordstyle=\color{blue},
    commentstyle=\color{green!60!black},
    stringstyle=\color{red},
    numberstyle=\tiny\color{gray},
    numbers=left,
    breaklines=true,
    frame=single,
    backgroundcolor=\color{gray!10}
}

% Julia language definition
\lstdefinelanguage{Julia}%
{morekeywords={abstract,break,case,catch,const,continue,do,else,elseif,%
    end,export,false,for,function,immutable,in,macro,module,otherwise,%
    quote,return,switch,true,try,type,typealias,using,while},%
sensitive,%
alsoother={$},%
morecomment=[l]\#,%
morecomment=[n]{\#=}{=\#},%
morestring=[s]{"}{"},%
morestring=[m]{'}{'},%
}[keywords,comments,strings]%

\title{Research Update: 11×11 Lattice with Distributed Load}
\author{Daniel Miles}
\date{November 15, 2025}

\begin{document}

\maketitle

\section{Introduction}

This update documents the extension of the mass-spring lattice simulation to an 11×11 system (121 masses) with a distributed load applied along the left edge, replacing the previous point load approach. This modification better simulates realistic loading conditions where forces are distributed across a structure rather than concentrated at a single point.

\section{System Extension: 11×11 Lattice}

The lattice size has been increased from 10×10 (100 masses) to 11×11 (121 masses), providing a larger domain for studying wave propagation and material response. The system maintains all previous features:
\begin{itemize}
    \item Exponential spring force law: $F = k \cdot (r/|r|) \cdot (e^{\alpha|r|} - 1)$
    \item Viscous damping on nearest-neighbor springs
    \item Nearest-neighbor and diagonal spring connections
    \item Immovable backplate constraint on the right side
    \item Column-based material property scaling
\end{itemize}

\section{Distributed Load Implementation}

\subsection{Force Distribution Pattern}

The distributed load is applied to all masses in column 1 (the left edge, $x=0$) with a trapezoidal distribution pattern:
\begin{equation}
F_i = \begin{cases}
F/20 & \text{if } i = 1 \text{ or } i = 11 \\
F/10 & \text{if } 2 \leq i \leq 10
\end{cases}
\end{equation}

\noindent where $F$ is the total force magnitude and $i$ is the row index (1 to 11). This pattern applies smaller forces at the edges (rows 1 and 11) and larger forces in the middle region (rows 2-10), creating a trapezoidal distribution that better represents typical distributed loading scenarios.

\subsection{Global Force Angle}

All distributed forces share a common direction specified by the global angle parameter $\theta$:
\begin{equation}
\mathbf{F}_i = F_i \begin{bmatrix} \cos\theta \\ \sin\theta \end{bmatrix}
\end{equation}

\noindent This allows the entire distributed load to be rotated while maintaining the relative distribution pattern. By default, $\theta = 0^\circ$ (horizontal to the right), but it can be configured to any angle.

\subsection{Implementation}

The distributed load is implemented in the ODE right-hand side function:

\begin{lstlisting}[language=Julia, caption=Distributed Load Application]
# --- DISTRIBUTED EXTERNAL DRIVING FORCE ON LEFT EDGE ---
if t <= F_ACTIVE_TIME
    fx_unit, fy_unit = calculate_force_components(1.0, FORCE_ANGLE_DEGREES)
    
    for i in 1:N  # All rows in column 1
        force_mag = get_distributed_force_magnitude(i)
        target_idx = lattice_idx(i, 1)  # Column 1, row i
        dvel[1, target_idx] += force_mag * fx_unit
        dvel[2, target_idx] += force_mag * fy_unit
    end
end
\end{lstlisting}

The work calculation sums contributions from all 11 distributed forces:

\begin{lstlisting}[language=Julia, caption=Work Calculation for Distributed Load]
function work_done_2d_distributed(sol)
    fx_unit, fy_unit = calculate_force_components(1.0, FORCE_ANGLE_DEGREES)
    w = 0.0
    for i in 1:N
        force_mag = get_distributed_force_magnitude(i)
        target_idx = lattice_idx(i, 1)
        # Calculate work for each force...
    end
    return w
end
\end{lstlisting}

\section{Comparison with Point Load}

The distributed load approach offers several advantages over the previous point load:

\begin{enumerate}
    \item \textbf{More realistic loading}: Better represents actual loading conditions in structures
    \item \textbf{Reduced stress concentrations}: Avoids the singular stress field that occurs at a point load
    \item \textbf{Better wave propagation}: Creates a more uniform initial excitation across the structure
    \item \textbf{Improved numerical stability}: Distributed forces reduce potential for localized instabilities
\end{enumerate}

\section{Visual Comparison}

\subsection{Force Distribution Pattern}

Figure~\ref{fig:force_distribution} illustrates the trapezoidal force distribution pattern applied to the 11 masses in column 1. The edges (rows 1 and 11) receive $F/20$ while the middle rows (2-10) receive $F/10$ each, creating a realistic distributed loading profile.

\begin{figure}[h]
    \centering
    \includegraphics[width=0.7\textwidth]{../figures/force_distribution_pattern.png}
    \caption{Force distribution pattern showing trapezoidal loading: $F/20$ at edges, $F/10$ in middle region.}
    \label{fig:force_distribution}
\end{figure}

\subsection{Point Load vs. Distributed Load}

Figure~\ref{fig:comparison} provides a visual comparison between the previous point load approach (10×10 system) and the new distributed load implementation (11×11 system). The distributed load creates a more uniform initial excitation across the left edge, avoiding the stress concentration that occurs with a single point load.

\begin{figure}[!h]
    \centering
    \begin{subfigure}[b]{0.48\textwidth}
        \centering
        \includegraphics[width=\textwidth]{../figures/distributed_load_11x11_snapshot.png}
        \caption{11×11 system with distributed load}
        \label{fig:distributed_load}
    \end{subfigure}
    \caption{11×11 lattice with distributed load showing 11 force arrows along the left edge. The snapshot is taken at $t = 0.2$ s during force application. The distributed load creates a more uniform initial excitation across the structure compared to a point load.}
    \label{fig:comparison}
\end{figure}

\subsection{Visualization Features}

The visualization has been updated to show all 11 force arrows simultaneously, providing clear indication of the distributed load pattern. Each arrow's length is proportional to the force magnitude at that location, making the trapezoidal distribution visually apparent. The color gradient across columns (light to dark blue) indicates the material property scaling, while the green force arrows clearly show the distributed loading pattern.

\section{Conclusions}

The extension to an 11×11 lattice with distributed loading represents a significant improvement in the simulation's capability to model realistic structural loading conditions. The trapezoidal force distribution pattern provides a more physically accurate representation of distributed loads while maintaining computational efficiency. The global angle parameter allows investigation of loading at various orientations, enhancing the system's versatility for studying different loading scenarios. \newline

\end{document}

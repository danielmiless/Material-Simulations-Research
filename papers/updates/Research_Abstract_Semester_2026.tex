\documentclass[11pt,a4paper]{article}
\usepackage[utf8]{inputenc}
\usepackage{hyperref}
\usepackage[margin=1in]{geometry}

\title{Research Abstract: Impact Resistance of Sandwich Structures}
\author{Daniel Miles\\Research Advisor: Dr. Romesh Batra}
\date{February 2, 2026}

\begin{document}

\maketitle

\begin{abstract}
\noindent This abstract outlines the research project on transient (time-dependent) deformations of sandwich structures subjected to impact loads such as those produced by a blast, a hurricane, or a collision of two high-speed moving deformable objects such as automobiles. The structures of interest consist of foam core sheets and fiber-reinforced composite face sheets. The objectives are to develop insights into the failure of these constituent materials and of the structure as a whole, thereby informing the design of impact-resistant protective systems. We hypothesize that the response of a structural element can be represented by linear and nonlinear springs and dashpots connected to each other, and we numerically solve the resulting coupled differential equations using in-house developed and verified software in the Julia programming language. The anticipated benefit of this work is to find the optimum structure configuration: the lay-up of foam layers and their thicknesses will be optimized using publicly available optimization algorithms to maximize, for a given areal mass density (mass per unit area of the impacted surface), the structure's impact resistance as measured by either the deflection of the non-impacted (back) face or the total force transmitted to a rigid substrate perfectly bonded to the back face. \newline

\noindent My role in this project will be to contribute to model development, numerical implementation of the spring-dashpot formulations in Julia, and optimization studies to identify optimal lay-up configurations. Expected end-of-semester outcomes include a validated numerical model for at least one impact scenario (e.g., blast or collision), an optimization framework for foam layer ordering and thickness, and a draft manuscript section summarizing the methodology and preliminary results. These outcomes represent working goals toward which I will direct my efforts during the semester.
\end{abstract}

\end{document}
